@proceedings{10.1115/GT2021-59580,
author = {Sanghera, Bhupinder Singh and Anand, Nitish and Souverein, Louis and Penin, Loïc and Pini, Matteo},
title = "{Adjoint-Based Optimisation of Rocket Engine Turbine Blades}",
volume = {Volume 2D: Turbomachinery — Multidisciplinary Design Approaches, Optimization, and Uncertainty Quantification; Radial Turbomachinery Aerodynamics; Unsteady Flows in Turbomachinery},
series = {Turbo Expo: Power for Land, Sea, and Air},
year = {2021},
month = {06},
abstract = "{Axial turbine stages for gas generator cycle type rocket engines typically employ highly-loaded supersonic stator vanes. Consequently, the flow pattern downstream of the vanes is characterized by shock waves which induce high-frequency excitation on the subsequent rotor. For this reason, the optimal design of the stator is crucial in the context of development of the next generation of high-performance rocket engines, where reusability is a principal design criterion. A thorough comprehension of the loss mechanisms combined with the adoption of automated optimisation techniques can therefore enable new stator designs that may provide large benefits in terms of overall turbine performance and lifespan.The scope of this study stems from these considerations and its objective is twofold, namely i) the shape optimisation of a supersonic stator for rocket engines and ii) the investigation of the loss mechanisms in supersonic axial turbine stator vanes at on- and off-design conditions. The investigation is performed on stator vanes that are under development for the first turbine stage of a gas generator cycle type rocket engine. The stator vanes are therefore optimised in order to reduce the profile losses by exploiting a novel adjoint optimisation framework for turbomachinery implemented in the open-source code SU2. The effect on the resulting flow field and loss sources is finally investigated.Results show that entropy based loss coefficient gains of 6\\% can be achieved via shape optimisation and that the fluid-dynamic performance of these vanes is less sensitive to changes in pressure ratio compared to the performance provided by the baseline configuration. Eventually, shock-waves remain the primary loss source.}",
doi = {10.1115/GT2021-59580},
url = {https://doi.org/10.1115/GT2021-59580},
note = {V02DT36A013},
eprint = {https://asmedigitalcollection.asme.org/GT/proceedings-pdf/GT2021/84935/V02DT36A013/6757415/v02dt36a013-gt2021-59580.pdf},
}


